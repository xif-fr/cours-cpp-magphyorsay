\documentclass[c]{beamer}
\usepackage{org-preamble}
\usepackage[cpp_teaching]{slide-style}
\usepackage{minted}
\newcommand{\inline}[1]{\mintinline[breaklines]{c++}{#1}}
\usepackage{xcolor}
\usepackage{newunicodechar}
\newcommand\Warning{%
 \makebox[1.4em][c]{%
 \makebox[0pt][c]{\raisebox{.1em}{\small!}}%
 \makebox[0pt][c]{\color{red}\Large$\bigtriangleup$}}}%
\newunicodechar{⚠}{\Warning}
\usetheme{default}
\title{Quelques spécificités du C++ (hors POO)}

\begin{document}

\maketitle
\def\theFancyVerbLine{%
  \color{white}\sffamily\tiny\arabic{FancyVerbLine}%
        {\tikz[remember picture,overlay]\node(minted-\arabic{FancyVerbLine}){};}%
}
\tikzset{codeblock/.style={color=#1!50,rounded corners=0.5ex, opacity=0.2, fill}}

%--------------------------------------------------------------------------------

\begin{frame}[fragile]{Spécifités notables du C++ par rapport au C}
  \begin{itemize}  
  \item Bibliothèque standard
  \item Déclaration des variables (emplacement libre)
  \item Arguments par défaut de fonction
  \item Surdéfinition de fonction
  \item Type \textit{bool}
  \item Espace de noms
  \item Références
  \item Allocation dynamique de mémoire : opérateur \textit{new} et \textit{delete}
  \item Exceptions
  \end{itemize}
\end{frame}

%------------------------------------------------------------------------------------

\begin{frame}[fragile]{Bibliothèque standard}
\begin{itemize}
\item Les bibliothèques du C ont été adaptées aux besoins du \Cpp :
\end{itemize}

\begin{columns}
\begin{column}{0.3\columnwidth}
\begin{minted}[fontsize=\footnotesize,samepage,mathescape,xrightmargin=0.5cm,xleftmargin=0.5cm]{c++}
// Inclusion en C
#include <math.h>
#include <time.h>
\end{minted}
\end{column}

\begin{column}{0.4\columnwidth}
\begin{minted}[fontsize=\footnotesize,samepage,mathescape,xrightmargin=0.5cm,xleftmargin=0.5cm]{c++}
// Inclusion en C++
#include <cmath>
#include <ctime>
\end{minted}
\end{column}
\end{columns}

\vspace{3em}
De nombreuses autres spécifiques au \Cpp existent (\textit{cf.} diapos 3), l'ensemble formant la bibliothèque standard du \Cpp (\textit{standard library}, ou \textit{std lib}).

\end{frame}

\begin{frame}[fragile]{Déclaration de variables, gestion de la mémoire}

\begin{itemize}
\item Possibilité de déclarer les variables locales quel que soit l'endroit
\end{itemize}
\begin{minted}[fontsize=\footnotesize,samepage,mathescape,xrightmargin=0.5cm,xleftmargin=0.5cm]{c++}
for (int i = 0 ; i < 10 ; i++) {
  double x = 1./i;
}
\end{minted}

\begin{cbox}[][][\centering]
⚠ \ La variable \texttt{x} n'existe plus en dehors de la boucle\\
\ \\
\ding{42} À garder en tête ! Notion de cycle de vie (\textit{scope}). Très différent de Python ou Java (\textit{garbage collector}).
\end{cbox}

\end{frame}

%--------------------------------------------------------------------------------

\begin{frame}[fragile]{Arguments par défaut de fonctions}
 \Cpp autorise l'utilisation de valeurs par défaut dans \structure{les prototypes de
  fonction}

\begin{minted}[fontsize=\footnotesize,samepage,mathescape,xrightmargin=0.5cm,xleftmargin=0.5cm]{c++}
// Prototype de fonction (déclaration)
void initialize(const double abs, const double ord = 3.4);

int main() {
  initialize(1.0);  // Utilisation
}

// Définition
void initialize(const double abs, const double ord)
{ ... }
\end{minted}

\begin{itemize}
\item En l'absence de second argument, le compilateur assigne la valeur 3.4 à la
variable \texttt{ord}

\item Les arguments concernés doivent \structure{obligatoirement} être les \structure{derniers} dans la
liste.
\end{itemize}
\end{frame}

%-----------------------------------------------------------------

\begin{frame}[fragile]{Surdéfinition (surcharge) de fonctions}
 En \Cpp deux fonctions peuvent porter le même nom dès lors qu'elles n'ont pas
les mêmes types d'argument

\begin{minted}[fontsize=\footnotesize,samepage,mathescape,xrightmargin=0.5cm,xleftmargin=0.5cm]{c++}
void dummy(int i, double a, char c);
void dummy(int i, int a, char c);
void dummy(int i);
\end{minted}

\pause
Le compilateur se charge d'appeler la "bonne" fonction par rapport au contexte
\emph{i.e.} au vu de la liste d'arguments donnée lors de l'appel.

\begin{minted}[fontsize=\footnotesize,samepage,mathescape,xrightmargin=0.5cm,xleftmargin=0.5cm]{c++}
int main()
{
  dummy(1, 1.0, 'a');
  dummy(1, 1, 'a');
  dummy(1);
}
\end{minted}
\end{frame}


\begin{comment}
%-----------------------------------------------------------------------
\begin{frame}[fragile]{Fonction \texttt{inline}}
 \structure{Les fonctions en ligne} remplacent (avantageusement) les macros utilisées en C
tout en conservant un temps d'exécution plus court

\vskip +10pt
On remplacera ainsi la déclaration en C
\begin{minted}[fontsize=\footnotesize,samepage,mathescape,xrightmargin=0.5cm,xleftmargin=0.5cm]{c++}
// Macro en C
#define CARRE(x) ((x) * (x))
\end{minted}
par
\begin{minted}[fontsize=\footnotesize,samepage,mathescape,xrightmargin=0.5cm,xleftmargin=0.5cm]{c++}
// Fonction en ligne en C++
inline double carre(const double x) { return x*x; }
\end{minted}
\end{frame}
%---------------------------------------------------------------------------
\end{comment}

\begin{frame}[fragile]{Le type \texttt{bool}}
Ce type est naturellement formé de deux valeurs notées \inline{true} et \inline{false} (remplaçant \texttt{1} et \texttt{0}).
\vspace{1em}

\begin{minted}[fontsize=\footnotesize,samepage,mathescape,xrightmargin=0.5cm,xleftmargin=0.5cm]{c++}
bool is_wrong = true;
if (is_wrong) {
  ...
}

bool is_nice = false;
bool is_ok = is_nice && !is_wrong;
\end{minted}
ou
\begin{minted}[fontsize=\footnotesize,samepage,mathescape,xrightmargin=0.5cm,xleftmargin=0.5cm]{c++}
bool is_ok = is_nice and not is_wrong;
\end{minted}

\end{frame}

%--------------------------------------------------------------------------------

\begin{frame}[fragile]{Les espaces de noms}
Les espaces de noms sont des zones de déclaration qui permettent de délimiter la recherche des noms des identificateurs par le compilateur.

\begin{minted}[fontsize=\footnotesize,samepage,mathescape,xrightmargin=0.5cm,xleftmargin=0.5cm]{c++}
namespace util {
  double angle (double re, double im) {
    /* Calcul l'angle d'un nombre complexe */
  }
}
namespace ondes {
  double angle (double phase1, double phase2) {
    /* Calcule l'angle entre deux phases, ramené dans ]-pi,+pi] */
  }
}
\end{minted}

Leur but est principalement d'éviter les conflits de noms entre plusieurs parties d'un projet :

\begin{minted}[fontsize=\footnotesize,samepage,mathescape,xrightmargin=0.5cm,xleftmargin=0.5cm]{c++}
util::angle(1, 0.4);
ondes::angle(pi/5, 9*pi/5);
\end{minted}

\end{frame}

\begin{frame}[fragile]{Les espaces de noms}

Toute la bibliothèque standard est dans l'espace \texttt{std}.
\vspace{1em}

\begin{minted}[fontsize=\footnotesize,samepage,mathescape,xrightmargin=0.5cm,xleftmargin=0.5cm]{c++}
#include <iostream>
/* using namespace std; */
...
std::cout << "Standard output" << std::endl;
...
\end{minted}

\vspace{1em}
Tout en permettant une syntaxe légère si désiré :
\vspace{1em}

\begin{minted}[fontsize=\footnotesize,samepage,mathescape,xrightmargin=0.5cm,xleftmargin=0.5cm]{c++}
#include <iostream>
using namespace std;
...
cout << "Standard output" << endl;
...
\end{minted}

\end{frame}

%--------------------------------------------------------------------------------

\begin{frame}[fragile]{Les espaces de noms}
 \begin{minted}[fontsize=\footnotesize,samepage,mathescape,xrightmargin=0.5cm,xleftmargin=0.5cm]{c++}
namespace utl
{
  void dump() { cout << "utl::dump" << endl; }
}
namespace io
{
  void dump() { cout << "io::dump" << endl; }
}
\end{minted}
\pause
\begin{minted}[fontsize=\footnotesize,samepage,mathescape,xrightmargin=0.5cm,xleftmargin=0.5cm]{c++}
// Utilisation de l'espace de nom io
using namespace io;

int main()
{
  // Par défaut, io::dump()
  dump();

  // Précision de l'espace de nom
  utl::dump();
  io::dump();
}
\end{minted}
\end{frame}

\begin{frame}[fragile]{Alias d'espace de noms}
\begin{minted}[fontsize=\footnotesize,samepage,mathescape,xrightmargin=0.5cm,xleftmargin=0.5cm]{c++}
#include <iostream>
 
namespace foo {
    namespace bar {
         namespace baz {
             int qux = 42;
         }
    }
}
 
namespace fbz = foo::bar::baz;
 
int main()
{
    std::cout << fbz::qux << '\n';
}
\end{minted}
\end{frame}

%--------------------------------------------------------------------------------

\part{Pointeurs, références et allocation dynamique}
\frame{\partpage}

%--------------------------------------------------------------------------------

\begin{frame}[fragile]{Rappels sur les adresses et pointeurs}
 \begin{itemize}
\item Tout objet manipulé par l'ordinateur est stocké en mémoire. Selon la nature de
l'objet, l'espace en mémoire alloué varie : par exemple, entier = 32 ou 64 bits

\item \structure{L'adresse} est l'endroit où se trouve la variable en mémoire. Elle s'obtient
via la syntaxe suivante : \texttt{\&NomDeLaVariable}

\item L'adresse n'étant ni plus ni moins qu'une valeur, on peut donc stocker cette
valeur dans une variable : \structure{un pointeur} est ainsi un conteneur d'adresse

\item Déclaration d'un pointeur :

\begin{minted}[fontsize=\footnotesize,samepage,mathescape,xrightmargin=0.5cm,xleftmargin=0.5cm]{c++}
int i = 10;  // variable
int * pt_i = &i;  // déclaration du pointeur et assign. à l'adresse de i
int j = *pt_i;  // dépointage et copie dans j
\end{minted}
\end{itemize}
\end{frame}

%---------------------------------------------------------------------------


\begin{frame}[fragile]{Notion de référence en \Cpp}
Le \Cpp introduit la notion de référence afin de faciliter la manipulation des variables

\begin{columns}
\begin{column}{0.3\columnwidth}
\begin{minted}[fontsize=\footnotesize,samepage,mathescape,xrightmargin=0.5cm,xleftmargin=0.5cm]{c++}
// Pointeur
int i = 10;
int * pt_i;
pt_i = &i;
(*pt_i)++;
\end{minted}
\end{column}

\begin{column}{0.6\columnwidth}
\begin{minted}[fontsize=\footnotesize,samepage,mathescape,xrightmargin=0.5cm,xleftmargin=0.5cm]{c++}
// Référence
int i = 10;
int & ref_i = i; // ref_i est une référence à i
ref_i++;
\end{minted}
\end{column}
\end{columns}

\vskip +10pt
\begin{itemize}
\item La déclaration d'une référence ne crée pas de nouvel objet, c'est simplement un alias vers une vraie variable. Il n'y a pas besoin de dépointer : utiliser une référence, c'est directement utiliser la variable référencée.

\item Toute référence doit se référer à un identificateur. Il est donc nécessaire \structure{d'initialiser} une référence : \texttt{int \& ref\_i;} ne compilera pas. Une référence pointe toujours vers le même objet, on ne peut pas changer sa destination.

\item Moins flexible, mais plus pratique et sûr. Suffisant dans la majorité des cas.
\end{itemize}
\end{frame}

\begin{frame}[fragile]{Références en tant qu'arguments de fonctions}
 \begin{columns}
\begin{column}{0.5\columnwidth}
\begin{cbox}[][lwuc](Transmission par adresse)
\begin{minted}[fontsize=\footnotesize,samepage,mathescape,xrightmargin=0.5cm,xleftmargin=0.5cm]{c++}
void echange (int * a, int * b) {
  int c = *a;
  *a = *b;
  *b = c;
}
...
int x = 10;
int y = 20;
echange(&x, &y);
\end{minted}

\pause
\end{cbox}
\end{column}

\begin{column}{0.5\columnwidth}
\begin{cbox}[][lwuc](Transmission par référence)
\begin{minted}[fontsize=\footnotesize,samepage,mathescape,xrightmargin=0.5cm,xleftmargin=0.5cm]{c++}
void echange (int & a, int & b) {
  int c = a;
  a = b;
  b = c;
}
...
int x = 10;
int y = 20;
echange(x, y);
\end{minted}
\end{cbox}
\end{column}
\end{columns}
\end{frame}


%--------------------------------------------------------------------------------

\begin{frame}[fragile]{Allocation dynamique de mémoire}
L'allocation dynamique de mémoire est nécessaire dès lors que la taille d'un objet, sa nature ou même son existence, n'est connue que lors de l'exécution du programme. Dans le cas d'un tableau de taille $n$, la déclaration

\begin{minted}[fontsize=\footnotesize,samepage,mathescape,xrightmargin=0.5cm,xleftmargin=0.5cm]{c++}
unsigned int n = 0;
std::cin >> n;
double tableau[n];
\end{minted}

n'est pas correcte du fait que le compilateur ne connait pas, au préalable, l'espace
mémoire nécessaire à l'allocation (statique).
\end{frame}

%-----------------------------------------------------------------------------------------

\begin{frame}[fragile]{Utilisation des opérateurs \texttt{new} et \texttt{delete}}
 \begin{itemize}
\item Pour rappel, en langage C, la gestion dynamique de mémoire fait appel aux
fonctions \texttt{malloc} et \texttt{free} (\texttt{stdlib.h})

\item \Cpp propose quatre nouveaux opérateurs :

\begin{itemize}
\item \inline{new} alloue une la quantité de mémoire nécessaie au type, et renvoie un pointeur sur la zone mémoire allouée.
\begin{minted}[fontsize=\footnotesize,samepage,mathescape,xrightmargin=0.5cm,xleftmargin=0.5cm]{c++}
double * ptr = new double;
*ptr = 3.7;
\end{minted}

\item \inline{delete} libère l'espace mémoire.
\begin{minted}[fontsize=\footnotesize,samepage,mathescape,xrightmargin=0.5cm,xleftmargin=0.5cm]{c++}
delete ptr;
// bien sûr, ne pas utiliser `ptr' après
\end{minted}

\item \inline{new[]} alloue la mémoire nécessaire pour un tableau de taille $n$ et renvoie un pointeur sur le début du tableau :
\begin{minted}[fontsize=\footnotesize,samepage,mathescape,xrightmargin=0.5cm,xleftmargin=0.5cm]{c++}
unsigned int n = 0;
std::cin >> n;
double * tableau = new double[n];
\end{minted}

\item \inline{delete[]} libère l'espace mémoire d'un pointeur-tableau :
\begin{minted}[fontsize=\footnotesize,samepage,mathescape,xrightmargin=0.5cm,xleftmargin=0.5cm]{c++}
delete[] tableau;
\end{minted}
\end{itemize}
\end{itemize}
\end{frame}


%----------------------------------------------------------------------------------------

\begin{frame}[fragile]{Portée et durée de vie des variables}
 \begin{columns}
\begin{column}{0.6\columnwidth}
\begin{cbox}[][lwuc][\footnotesize](Durée de vie limitée au bloc (ici boucle \texttt{for}))
\begin{minted}[fontsize=\footnotesize,samepage,mathescape,xrightmargin=0.5cm,xleftmargin=0.5cm]{c++}
for (int i = 0; i < 10; i++) {
  int k = 0;
  // À la fin du bloc,
  // destruction de k
 }
\end{minted}

\pause
\end{cbox}
\end{column}

\begin{column}{0.6\columnwidth}
\begin{cbox}[][lwuc][\footnotesize](Durée de vie indépendante du bloc)
\begin{minted}[fontsize=\footnotesize,samepage,mathescape,xrightmargin=0.5cm,xleftmargin=0.5cm]{c++}
for (int i = 0; i < 10; i++) {
  int * k = new int(0);
  // À la fin du bloc,
  // k existe en mémoire
 }
\end{minted}

\begin{cbox}[5][lrtuc][\centering][9][11]
Fuite de mémoire garantie
\end{cbox}
\end{cbox}
\end{column}
\end{columns}
\end{frame}

%--------------------------------------------------------------------------------------

\begin{frame}[fragile]{Portée et durée de vie des variables}
 \begin{columns}
\begin{column}{0.6\columnwidth}
\begin{cbox}[][lwuc][\footnotesize](Durée de vie limitée au bloc (ici boucle \texttt{for}))
\begin{minted}[fontsize=\footnotesize,samepage,mathescape,xrightmargin=0.5cm,xleftmargin=0.5cm]{c++}
for (int i = 0; i < 10; i++) {
  int k = 0;
  // À la fin du bloc,
  // destruction de k
 }
\end{minted}
\end{cbox}
\end{column}

\begin{column}{0.6\columnwidth}
\begin{cbox}[][lwuc][\footnotesize](Durée de vie indépendante du bloc)
\begin{minted}[fontsize=\footnotesize,samepage,mathescape,xrightmargin=0.5cm,xleftmargin=0.5cm]{c++}
for (int i = 0; i < 10; i++) {
  int * k = new int(0);
  ...
  delete k;
}
\end{minted}
\end{cbox}
\end{column}
\end{columns}
\end{frame}
\begin{frame}[fragile]{Portée et durée de vie des variables}
 \begin{columns}
\begin{column}{0.7\columnwidth}
\begin{cbox}[][lwuc][\footnotesize](Allocation sur la pile, “stack” (automatique))
\begin{minted}[fontsize=\footnotesize,samepage,mathescape,xrightmargin=0.5cm,xleftmargin=0.5cm]{c++}
int * pointeur_dix()
{
  int a = 10;
  return &a;
}

int main()
{
  int * pb = pointeur_dix();
  cout << *pb << endl;

  return 0;
}
\end{minted}
\end{cbox}
\end{column}

\begin{column}{0.6\columnwidth}
\begin{cbox}[][hidden][\footnotesize](Allocation sur le tas, “heap”)
\end{cbox}
\end{column}
\end{columns}
\end{frame}

\begin{frame}[fragile]{Portée et durée de vie des variables}
 \begin{columns}
\begin{column}{0.7\columnwidth}
\begin{cbox}[][lwuc][\footnotesize](Allocation sur la pile, “stack” (automatique))
\begin{minted}[fontsize=\footnotesize,samepage,mathescape,xrightmargin=0.5cm,xleftmargin=0.5cm]{c++}
int * pointeur_dix()
{
  int a = 10;
  return &a;
}

int main()
{
  int * pb = pointeur_dix();
  cout << *pb << endl;

  return 0;
}
\end{minted}

\begin{cbox}[5][lrtuc][\centering][3.5][7]
Le pointeur retourné contient une adresse obsolète
\end{cbox}

\pause
\end{cbox}
\end{column}

\begin{column}{0.6\columnwidth}
\begin{cbox}[][lwuc][\footnotesize](Allocation sur le tas, “heap”)
\begin{minted}[fontsize=\footnotesize,samepage,mathescape,xrightmargin=0.5cm,xleftmargin=0.5cm]{c++}
int * pointeur_dix()
{
  int * pa = new int(10);
  return pa;
}

int main()
{
  int * pb = pointeur_dix();
  cout << *pb << endl;
  delete pb;
  return 0;
}
\end{minted}
\end{cbox}
\end{column}
\end{columns}
\end{frame}

%--------------------------------------------------------------------------------

\begin{frame}[fragile]{Allocation dynamique de mémoire}
Ceci dit, en \Cpp moderne, la manipulation de pointeurs nus et l'allocation/libération manuelle de mémoire n'est \textbf{pas une bonne pratique}.\\

On préfèrera l'utilisation d'objets encapsulant et effectant automatiquement ces opérations $\rightarrow$ bibliothèque standard.\\
Et à défaut, on créera nos propres objets...
\end{frame}

%--------------------------------------------------------------------------------

\begin{frame}[fragile]{Principales notions en un exemple}
 \begin{itemize}
\item Le \Cpp introduit la notion de référence afin de faciliter la manipulation des
variables notamment au sein des fonctions

\begin{center}
\structure{\(\rightarrow\) transmission des arguments par référence}
\end{center}
\end{itemize}

\pause

\begin{itemize}
\item En \Cpp plusieurs fonctions peuvent porter le même nom dès lors qu'elles n'ont
pas le même contexte d'appel

\begin{center}
\structure{\(\rightarrow\) surdéfinition ou surcharge de fonction}
\end{center}
\end{itemize}

\pause

\begin{itemize}
\item Les macros sont à éviter. Les seules véritables directives de préprocesseur à
utiliser sont \structure{\texttt{\#include}, \texttt{\#ifndef} / \texttt{\#define}}

\begin{center}
\structure{\(\rightarrow\) compilation séparée}
\end{center}
\end{itemize}
\end{frame}


%--------------------------------------------------------------------


\begin{frame}[fragile]{}
 \begin{cbox}[][lwuc](\texttt{dummy.h})
\begin{minted}[linenos,firstnumber=1,fontsize=\footnotesize,samepage,mathescape,xrightmargin=0.5cm,xleftmargin=0.5cm]{c++}
#ifndef __dummy_h__
#define __dummy_h__ 1
void dummy();
void dummy(int i);
void dummy(int & i, double & a, char & c);
#endif
\end{minted}

\vspace{-0cm}
\end{cbox}

\begin{cbox}[][lwuc](\texttt{dummy.cpp})
\begin{minted}[linenos,firstnumber=1,fontsize=\footnotesize,samepage,mathescape,xrightmargin=0.5cm,xleftmargin=0.5cm]{c++}
#include<iostream>
#include "dummy.h"
using namespace std;
void dummy() {cout << "fct type void, pas d'argument" << endl;}
void dummy(int i) {
  cout << "fct type void, 1 argument type entier" << endl;}
void dummy(int & i, double & a, char & c) {
  cout << "fct type void, 3 arguments : ref sur entier, réel et caractère"
  << endl;}
\end{minted}

\vspace{-0cm}
\end{cbox}
\end{frame}

%------------------------------------------------------

\begin{frame}[fragile]{}
\begin{cbox}[][lwuc](\texttt{test\_dummy.cpp})
  \begin{minted}[linenos,firstnumber=1,fontsize=\footnotesize,samepage,mathescape,xrightmargin=0.5cm,xleftmargin=0.5cm]{c++}
#include<iostream>
using namespace std;
#include "dummy.h"
int main()
{
  cout << "Programme test des fonctions dummy" << endl;
  dummy();
  int i = 10; double  d = 2.0; char c = 'a';
  dummy(i, d, c);
}
\end{minted}
\end{cbox}
\end{frame}


%----------------------------------------------------------------------------------------

\begin{frame}[fragile]{}
 \begin{cbox}[6][lbtuc][\centering][9][1]
Transmission par référence
\end{cbox}

\begin{cbox}[][lwuc](\texttt{dummy.h})
\begin{minted}[linenos,firstnumber=1,fontsize=\footnotesize,samepage,mathescape,xrightmargin=0.5cm,xleftmargin=0.5cm]{c++}
#ifndef __dummy_h__
#define __dummy_h__ 1
void dummy();
void dummy(int i);
void dummy(int & i, double & a, char & c);
#endif
\end{minted}
\begin{tikzpicture}[remember picture,overlay]
  \draw[codeblock=blue]
  ([yshift=-0.75ex,xshift= 2ex]minted-5) rectangle
  ([yshift=+1.50ex,xshift=43ex]minted-5);
  \node[] () [xshift=+6.6cm, yshift=+0.4ex, right=of minted-5.east]
       {\textcolor{blue}{\scriptsize Déclaration de la fonction}};
\end{tikzpicture}

\vspace{-0.7cm}
\end{cbox}

\begin{cbox}[][lwuc](\texttt{dummy.cpp})
\begin{minted}[linenos,firstnumber=1,fontsize=\footnotesize,samepage,mathescape,xrightmargin=0.5cm,xleftmargin=0.5cm]{c++}
...
#include "dummy.h"
void dummy() {...}
void dummy(int i) {...}
void dummy(int & i, double & a, char & c) {...}
\end{minted}
\begin{tikzpicture}[remember picture,overlay]
  \draw[codeblock=blue]
  ([yshift=-0.75ex,xshift= 2ex]minted-4) rectangle
  ([yshift=+1.50ex,xshift=43ex]minted-4);
  \node[] () [xshift=+6.6cm, yshift=+0.4ex, right=of minted-4.east]
       {\textcolor{blue}{\scriptsize Définition de la fonction}};
\end{tikzpicture}

\vspace{-0.7cm}
\end{cbox}

\begin{cbox}[][lwuc](\texttt{test\_dummy.cpp})
\begin{minted}[linenos,firstnumber=1,fontsize=\footnotesize,samepage,mathescape,xrightmargin=0.5cm,xleftmargin=0.5cm]{c++}
#include "dummy.h"
int main()
{
  dummy();
  int i = 10; double d = 2.0; char c = 'a';
  dummy(i, d, c);
}
\end{minted}
\begin{tikzpicture}[remember picture,overlay]
  \draw[codeblock=blue]
  ([yshift=-0.75ex,xshift= 2ex]minted-6) rectangle
  ([yshift=+1.50ex,xshift=43ex]minted-6);
  \node[] () [xshift=+6.6cm, yshift=+0.4ex, right=of minted-6.east]
       {\textcolor{blue}{\scriptsize Appel de la fonction}};
\end{tikzpicture}
\end{cbox}
\end{frame}

\begin{frame}[fragile]{}
 \begin{cbox}[6][lgtuc][\centering][9][1]
Surdéfinition de fonction
\end{cbox}

\begin{cbox}[][lwuc](\texttt{dummy.h})
\begin{minted}[linenos,firstnumber=1,fontsize=\footnotesize,samepage,mathescape,xrightmargin=0.5cm,xleftmargin=0.5cm]{c++}
#ifndef __dummy_h__
#define __dummy_h__ 1
void dummy();
void dummy(int i);
void dummy(int & i, double & a, char & c);
#endif
\end{minted}
\begin{tikzpicture}[remember picture,overlay]
  \draw[codeblock=green]
  ([yshift=-0.75ex,xshift= 2ex]minted-5) rectangle
  ([yshift=+6.00ex,xshift=43ex]minted-5);
  \node[] () [xshift=+6.6cm, yshift=+0.4ex, right=of minted-4.east]
       {\textcolor{green}{\scriptsize Déclaration des fonctions}};
\end{tikzpicture}

\vspace{-0.7cm}
\end{cbox}

\begin{cbox}[][lwuc](\texttt{dummy.cpp})
\begin{minted}[linenos,firstnumber=1,fontsize=\footnotesize,samepage,mathescape,xrightmargin=0.5cm,xleftmargin=0.5cm]{c++}
...
#include "dummy.h"
void dummy() {...}
void dummy(int i) {...}
void dummy(int & i, double & a, char & c) {...}
\end{minted}
\begin{tikzpicture}[remember picture,overlay]
  \draw[codeblock=green]
  ([yshift=-0.75ex,xshift= 2ex]minted-4) rectangle
  ([yshift=+6.00ex,xshift=43ex]minted-4);
  \node[] () [xshift=+6.6cm, yshift=+0.4ex, right=of minted-3.east]
       {\textcolor{green}{\scriptsize Définition des fonctions}};
\end{tikzpicture}

\vspace{-0.7cm}
\end{cbox}

\begin{cbox}[][lwuc](\texttt{test\_dummy.cpp})
\begin{minted}[linenos,firstnumber=1,fontsize=\footnotesize,samepage,mathescape,xrightmargin=0.5cm,xleftmargin=0.5cm]{c++}
#include "dummy.h"
int main()
{
  dummy();
  int i = 10; double d = 2.0; char c = 'a';
  dummy(i, d, c);
}
\end{minted}
\begin{tikzpicture}[remember picture,overlay]
  \draw[codeblock=green]
  ([yshift=-0.75ex,xshift= 2ex]minted-4) rectangle
  ([yshift=+1.50ex,xshift=43ex]minted-4);
  \draw[codeblock=green]
  ([yshift=-0.75ex,xshift= 2ex]minted-6) rectangle
  ([yshift=+1.50ex,xshift=43ex]minted-6);
  \node[] (t) [xshift=+6.6cm, yshift=+0.4ex, right=of minted-5.east]{
    \textcolor{green}{\scriptsize Appel des fonctions}};
  \draw[->, green] (t.west) to [out=180, in=0]
  ([xshift=42.5ex, yshift=+0.4ex]minted-4.east);
  \draw[->, green] (t.west) to [out=180, in=0]
  ([xshift=42.5ex, yshift=+0.4ex]minted-6.east);
\end{tikzpicture}
\end{cbox}
\end{frame}

\begin{frame}[fragile]{}
 \begin{cbox}[6][lotuc][\centering][9][14]
Directives de préprocesseur
\end{cbox}

\begin{cbox}[][lwuc](\texttt{dummy.h})
\begin{minted}[linenos,firstnumber=1,fontsize=\footnotesize,samepage,mathescape,xrightmargin=0.5cm,xleftmargin=0.5cm]{c++}
#ifndef __dummy_h__
#define __dummy_h__ 1
void dummy();
void dummy(int i);
void dummy(int & i, double & a, char & c);
#endif
\end{minted}
\begin{tikzpicture}[remember picture,overlay]
  \draw[codeblock=orange]
  ([yshift=-0.75ex,xshift= 2ex]minted-2) rectangle
  ([yshift=+3.75ex,xshift=43ex]minted-2);
  \draw[codeblock=orange]
  ([yshift=-0.75ex,xshift= 2ex]minted-6) rectangle
  ([yshift=+1.50ex,xshift=43ex]minted-6);
\end{tikzpicture}

\vspace{-0.7cm}
\end{cbox}

\begin{cbox}[][lwuc](\texttt{dummy.cpp})
\begin{minted}[linenos,firstnumber=1,fontsize=\footnotesize,samepage,mathescape,xrightmargin=0.5cm,xleftmargin=0.5cm]{c++}
...
#include "dummy.h"
void dummy() {...}
void dummy(int i) {...}
void dummy(int & i, double & a, char & c) {...}
\end{minted}
\begin{tikzpicture}[remember picture,overlay]
  \draw[codeblock=orange]
  ([yshift=-0.75ex,xshift= 2ex]minted-1) rectangle
  ([yshift=+1.50ex,xshift=43ex]minted-1);
\end{tikzpicture}

\vspace{-0.7cm}
\end{cbox}

\begin{cbox}[][lwuc](\texttt{test\_dummy.cpp})
\begin{minted}[linenos,firstnumber=1,fontsize=\footnotesize,samepage,mathescape,xrightmargin=0.5cm,xleftmargin=0.5cm]{c++}
#include "dummy.h"
int main()
{
  dummy();
  int i = 10; double d = 2.0; char c = 'a';
  dummy(i, d, c);
}
\end{minted}
\begin{tikzpicture}[remember picture,overlay]
  \draw[codeblock=orange]
  ([yshift=-0.75ex,xshift= 2ex]minted-1) rectangle
  ([yshift=+1.50ex,xshift=43ex]minted-1);
\end{tikzpicture}
\end{cbox}
\end{frame}

%----------------------------------------------------------------

\begin{frame}{Compilation}
\begin{prompt}
g++ dummy.cpp test\_dummy.cpp -o test\_dummy.exe
\end{prompt}

\begin{cbox}[][][\centering]
\ding{42} Les fichiers d'en-tête ne sont jamais compilés !
\end{cbox}
\end{frame}

%----------------------------------------------------------------

\part{Objets constants}
\frame{\partpage}

%----------------------------------------------------------------

\begin{frame}[fragile]{Objets constants}
Le mot-clé \inline{const} interdit la modification de l'objet auquel il s'applique.
\vspace{1em}

\begin{minted}[fontsize=\footnotesize,samepage,mathescape,xrightmargin=0.5cm,xleftmargin=0.5cm]{c++}
const int i = 42;
i = 3;  // erreur de compilation
\end{minted}

\vspace{1em}
\pause

Deux utilités :
\begin{itemize}
  \item Protège l'objet d'une modification (contrainte imposée à soi-même et aux autres utilisateurs du code)
  \item Indique au compilateur qu'il peut effectuer des optimisations supplémentaires
\end{itemize}

\end{frame}

%----------------------------------------------------------------

\begin{frame}[fragile]{Pointeur constant ou pointeur sur constante ?}
Petite subtilité :
\begin{itemize}
  \item \inline{char * const str;} est un \textit{pointeur constant} sur un objet de type \inline{char} mutable\\
  $\Rightarrow$ \inline{str[0]='a'} est légal, mais \inline{str=NULL} est illégal.
  \item \inline{const char * str;} est un pointeur (mutable) sur un objet de type \inline{const char}, ie. une constante\\
  $\Rightarrow$ \inline{str[0]='a';} est illégal, mais \inline{str=NULL;} est légal.
\end{itemize}
\vspace{1em}
\pause
La question ne se pose pas pour les références, qui sont nécessairement constantes. Seule la forme \inline{const type&} a une utilité, pour empêcher l'objet référencé d'être modifié.
\end{frame}

%----------------------------------------------------------------

\begin{frame}[fragile]{Rappel : passage par valeur, par référence}
 \begin{columns}
\begin{column}{0.5\columnwidth}
\begin{cbox}[][lwuc][\footnotesize](Passage par valeur)
\begin{minted}[fontsize=\footnotesize,samepage,mathescape,xrightmargin=0.5cm,xleftmargin=0.5cm]{c++}
void f (int i) {
  i = 10;
  cout << "f : " << i << endl;
}

void main () {
  int e = 0;
  f(e);
  cout << "main : " << e << endl;
}
\end{minted}

\onslide<2>\tt
\begin{cbox}
f : 10\\
main : 0
\end{cbox}

\onslide<all>
\end{cbox}
\end{column}
\begin{column}{0.5\columnwidth}
\begin{cbox}[][lwuc][\footnotesize](Passage par référence)
\begin{minted}[fontsize=\footnotesize,samepage,mathescape,xrightmargin=0.5cm,xleftmargin=0.5cm]{c++}
void g (int& i) {
  i = 10;
  cout << "g : " << i << endl;
}

void main () {
  int e = 0;
  g(e);
  cout << "main : " << e << endl;
}
\end{minted}

\onslide<2>\tt
\begin{cbox}
g : 10\\
main : 10
\end{cbox}
\end{cbox}
\end{column}
\end{columns}
\end{frame}

%-------------------------------------

\begin{frame}[fragile]{Passage d'objets par valeur, par référence constante}
 \onslide<all>
\begin{columns}
\begin{column}{0.5\columnwidth}
\begin{cbox}[][lwuc][\footnotesize](Passage par valeur)
\begin{minted}[fontsize=\footnotesize,samepage,mathescape,xrightmargin=0.5cm,xleftmargin=0.5cm]{c++}
double somme (BigData data) {
  double s = 0;
  for (...)
    s += data[i];
  return s;
}

BigData test ("un_fichier.csv");
cout << somme(test) << endl;
\end{minted}

\onslide<2>
\begin{cbox}[][][\centering]
L'objet est \textbf{copié}\\(constructeur par copie appelé)
\end{cbox}
\vspace{0em}

\onslide<all>
\end{cbox}
\end{column}
\begin{column}{0.5\columnwidth}
\begin{cbox}[][lwuc][\footnotesize](Passage par référence constante)
\begin{minted}[linenos,firstnumber=1,fontsize=\footnotesize,samepage,mathescape,xrightmargin=0.5cm,xleftmargin=0.5cm]{c++}
double somme (const BigData& data) {
  double s = 0;
  for (...)
    s += data[i];
  return s;
}

BigData test ("un_fichier.csv");
cout << somme(test) << endl;
\end{minted}
\begin{tikzpicture}[remember picture,overlay]
  \draw[codeblock=blue]
  ([yshift=-0.75ex,xshift=17ex]minted-1) rectangle
  ([yshift=+1.75ex,xshift=37ex]minted-1);
\end{tikzpicture}
\vspace{-1.6em}
\onslide<2>
\begin{cbox}[][][\centering]
Une copie inutile est évitée, mais l'objet \textbf{reste protégé} d'éventuelles modification.
\end{cbox}
\end{cbox}
\end{column}
\end{columns}

\vspace{1em}
\pause
$\rightarrow$ Utile pour les objets volumineux (tableaux, listes...). Parfois indispensable (objet non copiable).

\end{frame}

%-------------------------------

\begin{frame}[fragile]{Copie en \Cpp}

En \Cpp, le comportement par défaut lors d'un appel ou d'une assignation est la copie (profonde). C'est \structure{à vous} de choisir entre la \textit{copie}, le \textit{passage par référence constante} (pas de modifications), ou le \textit{passage par référence} simple (pour effectuer des modifications)\\

\pause
Différent de Python, où les objets composites ne sont pas copiés en profondeur :
\begin{minted}[fontsize=\footnotesize,samepage,mathescape,xrightmargin=0.5cm,xleftmargin=0.5cm]{python}
def test (liste, nombre):
  liste[2] = 0
  nombre = 0

liste = [1,2,3]
nombre = 4
somme(liste, nombre)
print(liste, nombre)
\end{minted}
\onslide<3>
\begin{cbox}
\texttt{[1,2,0] 4}
\end{cbox}

\end{frame}


\end{document}
