\documentclass[c]{beamer}
                \usepackage{org-preamble}
                \usepackage[cpp_teaching]{slide-style}
                \usepackage{minted}
\usetheme{default}

\title{Programation orientée objet - Classes en C++}

\begin{document}

\maketitle

%------------------------------------------------------------------

\begin{frame}[fragile]{Plein de tableaux...}

\begin{minted}[fontsize=\small]{c++}
  vector<string> prénoms_étudiants = {"A", "B", "C"};
  vector<string> noms_étudiants = {"D", "E", "F", "G"};
  vector<date> datesnaissance_étudiants = {date(2022,10,4), ...};
  vector<float> notes_étudiants (4);
  notes_étudiants[2] = 19;
\end{minted}

\end{frame}

\begin{frame}[fragile]{Qu'a-t-on envie de faire ?}

\pause
$\rightarrow$ Regrouper dans une "boite" / structure
\begin{minted}[fontsize=\small]{c++}
struct Étudiant {
  string prénom;
  string nom;
  date date_naissance;
  float note;
};

vector<Étudiant> étudiants = {
  { "A", "D", date(2022,10,4), 0 },
  ...
};
étudiants[2].note = 19;
\end{minted}

\end{frame}

%------------------------------------------------------------------

\begin{frame}[fragile]{Un code infernal}

  \begin{columns}[t]
    \hspace{-0.8cm} 
    \begin{column}{0.6\textwidth}
      \begin{minted}[fontsize=\footnotesize,samepage,mathescape,xrightmargin=0.5cm,xleftmargin=0.5cm]{c++}
// appareil A
canal_comm_a = 192.168.1.5:3000
login(canal_comm_a, "utilisateur", "mdp")
calibration_a = 0.74 // W/A
dernière_commande = 0

// appareils B
port_usb = "/dev/usb5"
p = open(port_usb, 115200)
voltage = +inf
// ... deux autres appareils comme B ...

wait(5 sec) // attendre que A démarre
envoyer(canal_comm_a, "initialize, 42")
wait(1 sec)
        \end{minted}
    \end{column}
    \hspace{-0.6cm} 
    \begin{column}{0.6\textwidth}
      \begin{minted}[fontsize=\footnotesize,samepage,mathescape,xrightmargin=0.5cm,xleftmargin=0.5cm]{c++}
while (voltage > 0) {
  
  char buffer[100]
  voltage = atof( read(p, buffer, 100) )
  if (voltage > 16) afficher erreur "..."
  r = write(p, "reset")
  if (r < 0) afficher erreur "..."
  // ... deux autres appareils comme B ...

  puissance = ...  // faire des calculs

  I = puissance*calibration_a
  envoyer(canal_comm_a, "setcurr {I:.4e} 0")
  dernière_commande = I
  envoyer(canal_comm_a, "bidule")
  envoyer(canal_comm_a, "chose")
  état = envoyer(canal_comm_a, "currstate")
  if (état > 4 or état < 0)
    afficher erreur "..."
}
        \end{minted}
    \end{column}
  \end{columns}
\end{frame}

%------------------

\begin{frame}[fragile]{Qu'a-t-on envie de faire ?}

Fonction du programme noyée dans les détails techniques, redondance...

\end{frame}

\begin{frame}[fragile]{Qu'a-t-on envie de faire ?}

\pause
Cacher, "emballer" :
\begin{itemize}[<+->]
  \item les routines spécifiques à chaque appareil $\rightarrow$ faire des \emph{fonctions} pour dédupliquer \& regrouper 
  \item les cannaux de communcations, calibrations, états... $\rightarrow$ dans une boite / \emph{structure}
  \item combiner ces deux idées en une \emph{\textbf{interface}} claire et simple
\end{itemize}

\end{frame}

\begin{frame}[fragile]{Idéalement}

\begin{minted}[fontsize=\footnotesize,samepage,mathescape,xrightmargin=0.5cm,xleftmargin=0.5cm]{c++}
// appareil A
appareil_a = AppareilTypeA(192.168.1.5, "utilisateur", "mdp", 0.74 /*W/A*/)

// appareils B
appareil_b1 = AppareilTypeB("/dev/usb5")
appareil_b2 = AppareilTypeB("/dev/usb7")
appareil_b3 = AppareilTypeB("/dev/usb8")

appareil_a.wait_and_initialize()

while (appareil_b1.last_voltage > 0) {
  
  v1 = appareil_b1.read_voltage()
  v2 = appareil_b2.read_voltage()
  v3 = appareil_b3.read_voltage()

  puissance = ...  // faire des calculs

  appareil_a.set_power(puissance)
  if (appareil_a.état > 4 or appareil_a.état < 0)
    afficher erreur "..."
}
\end{minted}
\end{frame}

%------------------------------------------------------------------

\begin{frame}[fragile]{Notion de classe}
 \begin{itemize}[<+->]
\item Programmation orientée objet : analyse du problème en termes \structure{\emph{de nature et de structure des
données}}, et des \structure{\emph{actions}} sur ces structures

\item \(\neq\) programmation impérative (pensée en terme de procédure, d'instructions)

\item Une classe est

\begin{enumerate}
\item \structure{une généralisation de la notion de type (\texttt{int}, \texttt{double},\ldots{})}

\item \structure{une association de \textit{\textcolor{magenta}{données membres}}
  ($\equiv$ \textit{\textcolor{magenta}{données}} $\equiv$ \textit{\textcolor{magenta}{membres}} $\equiv$ \textit{\textcolor{magenta}{attributs}} ) et d'opérations
  s'y rapportant, les \textit{\textcolor{magenta}{méthodes}} ($\equiv$ \textit{\textcolor{magenta}{fonctions membres}})}
\end{enumerate}

\item Mais la POO, c'est bien plus que ça...
\end{itemize}
\end{frame}

%------------------------------------------------------------------

\begin{frame}[fragile]{Déclaration de classe}
 \begin{itemize}
\item Classe \structure{\(\rightarrow\) généralisation de la notion de type}
\begin{cbox}[][lwuc](\texttt{particule.h})
\begin{minted}[fontsize=\footnotesize,samepage,mathescape,xrightmargin=0.5cm,xleftmargin=0.5cm]{c++}

class particule
{
  public:  // visibilité des membres (cf. cours encapsulation)

  // Déclaration de données membres
  double masse; // kg
  double charge; // C
  vec3 impulsion; // kg·m/s
};
\end{minted}
\end{cbox}
\end{itemize}

\pause

\begin{cbox}[][][\centering]
\ding{42} Ne pas oublier le point virgule à la fin de la déclaration
\end{cbox}
\end{frame}

%--------------------------------------------------------------

\begin{frame}[fragile]{Déclaration de classe}
 \begin{itemize}
\item Classe \structure{\(\rightarrow\) association de données membres et de méthodes (fonctions
membres)}
\begin{cbox}[][lwuc](\texttt{particule.h})
  \begin{minted}[fontsize=\footnotesize,samepage,mathescape,xrightmargin=0.5cm,xleftmargin=0.5cm]{c++}
class particule
{
  public:

  // Déclaration de données membres
  double masse; // kg
  double charge; // C
  vec3 impulsion; // kg·m/s

  // Déclaration des méthodes
  void initialise(double masse_MeV, double charge_u, vec3 impulsion_SI);
  void afficher();
  double masse_en_MeV();
  double énergie_en_MeV();
};
\end{minted}
\end{cbox}
\end{itemize}
\end{frame}

%-------------------------------------------

\begin{frame}[fragile]{Déclaration de classe}
 \begin{itemize}
\item Classe \structure{\(\rightarrow\) association de données membres et de méthodes}
\begin{cbox}[][lwuc](\texttt{particule.h})
  \begin{minted}[fontsize=\footnotesize,samepage,mathescape,xrightmargin=0.5cm,xleftmargin=0.5cm]{c++}
#ifndef PARTICULE_H
#define PARTICULE_H    
class particule
{
  public:

  // Déclaration de données membres
  double masse; // kg
  double charge; // C
  vec3 impulsion; // kg·m/s

  // Déclaration des méthodes
  void initialise(double masse_MeV, int charge_u, vec3 impulsion_SI);
  void afficher();
  double masse_en_MeV();
  double énergie_en_MeV();
};
#endif
\end{minted}
\end{cbox}
\end{itemize}
\end{frame}

%-------------------------------------------
 
\begin{frame}[fragile,label={sec:orgheadline5}]{Définition de méthodes de classes}
 \begin{itemize}
\item Identique à la définition d'une fonction mais préfixée du nom de la classe et
de l'opérateur de portée \texttt{::}
\begin{cbox}[][lwuc](\texttt{particule.cpp})
\begin{minted}[fontsize=\footnotesize,samepage,mathescape,xrightmargin=0.5cm,xleftmargin=-1.5cm]{c++}
#include <iostream>
#include "particule.h"

void particule::initialise(double masse_MeV, int charge_u, vec3 impulsion_SI) {
  masse = masse_MeV * J_per_MeV / (c*c);
  charge = charge_u * unit_charge;
  impulsion = impulsion_SI;
}

void particule::afficher() {
  std::cout << "m=" << masse_en_MeV() << " MeV/c², q=" << charge/unit_charge << ...
}

double particule::énergie_en_MeV() { // E² = m²c⁴ + p²c²
  c2 = c * c;
  return sqrt( pow(masse*c2, 2) + impulsion.norme_carrée()*c2 ) / J_per_MeV;
}
\end{minted}
\end{cbox}
\end{itemize}
\end{frame}

%----------------------------------------------------------------------------

\begin{frame}[fragile,label={sec:orgheadline6}]{Exemple d'utilisation de la classe \texttt{particule}}
\begin{cbox}[][lwuc](\texttt{test\_particule.cpp})
\begin{minted}[fontsize=\footnotesize,samepage,mathescape,xrightmargin=0.5cm,xleftmargin=0.5cm]{c++}
#include<iostream>
using namespace std;
#include "particule.h"

int main()
{
  // Instanciation d'objets de type particule
  particule my_electron, my_proton;
  my_electron.initialise(0.511, -1, {2e-23,-4e-23,0});
  my_proton.initialise(938.0, +1, {0,0,0});

  cout << "my_electron :" << endl;
  my_electron.afficher();
  double E = my_electron.énergie_en_MeV();
  cout << "énergie en MeV = " << E << endl;
}
\end{minted}
\end{cbox}
\end{frame}

%---------------------------------------------------------

\begin{frame}[fragile,label={sec:orgheadline7}]{Exemple d'utilisation de la classe \texttt{particule} : pointeurs}
  \begin{cbox}[][lwuc](\texttt{test\_particule.cpp})
 \begin{minted}[fontsize=\footnotesize,samepage,mathescape,xrightmargin=0.5cm,xleftmargin=0.5cm]{c++}
#include<iostream>
using namespace std;
#include "particule.h"

int main()
{
  // Instanciation d'objets de type particule
  particule* ptr_electron = new particule;
  ptr_electron->initialise(0.511, -1, {2e-23,-4e-23,0});

  cout << "my_electron :" << endl;
  my_electron->afficher();
  double E = my_electron->énergie_en_MeV();
  cout << "énergie en MeV = " << E << endl;
 
  delete ptr_electron;
}
 \end{minted}
 \end{cbox}
\end{frame}


\end{document}
