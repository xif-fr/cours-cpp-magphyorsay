\documentclass[c]{beamer}
                \usepackage{org-preamble}
                \usepackage[cpp_teaching]{slide-style}
\usetheme{default}

\title{Patrons de fonctions et de classes}

\begin{document}

\maketitle

\begin{frame}[fragile,label={sec:orgheadline1}]{Notions de patrons ou \texttt{template}}
 \begin{itemize}
\item \structure{la surdéfinition de fonctions} permet de donner un nom unique à plusieurs
fonctions dont le contexte d'appel, c'est-à-dire les arguments de la fonction,
diffère,
\end{itemize}

\pause

\begin{itemize}
\item \structure{la redéfinition} intervient entre classes héritées et permet de spécialiser
une fonction membre suivant les buts de la classe dérivée,
\end{itemize}

\pause

\begin{itemize}
\item \structure{les patrons ou \texttt{template} de fonction} permettent d'écrire une seule fois la
définition d'une fonction afin que le compilateur puisse l'adapter au type des
arguments d'entrée ou de valeur de retour.
\end{itemize}
\end{frame}

\begin{frame}[fragile,label={sec:orgheadline2}]{Illustration de l'utilisation de fonction patron}
 \begin{minted}[fontsize=\footnotesize,samepage,mathescape,xrightmargin=0.5cm,xleftmargin=0.5cm]{c++}
int min(const int a_, const int b_)
{
  return a_ < b_ ? a_ : b_;
}

float min(const float a_, const float b_)
{
  return a_ < b_ ? a_ : b_;
}

char min(const char a_, const char b_)
{
  return a_ < b_ ? a_ : b_;
}
\end{minted}
\end{frame}

\begin{frame}[fragile,label={sec:orgheadline3}]{Illustration de l'utilisation de fonction patron}
 \begin{minted}[fontsize=\footnotesize,samepage,mathescape,xrightmargin=0.5cm,xleftmargin=0.5cm]{c++}
template<typename T> T min(const T & a_, const T & b_)
{
  return a_ < b_ ? a_ : b_;
}
\end{minted}

\begin{itemize}
\item lors de la compilation, suivant le type des arguments fournis, le compilateur
\structure{enregistre} c'est-à-dire implémente le mécanisme adéquat
\end{itemize}

\pause
\begin{minted}[fontsize=\footnotesize,samepage,mathescape,xrightmargin=0.5cm,xleftmargin=0.5cm]{c++}
int main()
{
  int i1 = 2, i2 = 7;
  float f1 = 3.4, f2 = 5.6;
  char c1 = 'd', c2 = 'z';

  cout << "min(i1,i2)   = " << min(i1,i2) << endl;
  cout << "min(f1,f2)   = " << min(f1,f2) << endl;
  cout << "min(c1,c2)   = " << min(c1,c2) << endl;
}
\end{minted}
%cout << "min(&c1,&c2) = " << min(&c1, &c2) << endl; 
\end{frame}

\begin{frame}[fragile,label={sec:orgheadline4}]{Illustration de l'utilisation de fonction patron}
 \begin{itemize}
\item \Cpp permettant la surcharge des opérateurs, la fonction patron \texttt{min} peut
donc s'utiliser avec des objets plus complexes
\end{itemize}

\begin{minted}[fontsize=\footnotesize,samepage,mathescape,xrightmargin=0.5cm,xleftmargin=0.5cm]{c++}
class point {
public:
  double norme() const;
private:
  double m_x;
  double m_y;
};

bool operator<(const point & p1_, const point & p2_)
{
  return p1_.norme() < p2_.norme();
}
...
point my_point1(1,1), my_point2(2,2);
cout << "min(p1,p2) =" << min(my_point1, my_point2) << endl;
\end{minted}
\end{frame}

\begin{frame}[fragile,label={sec:orgheadline5}]{Utilisation de patron de classe}
 La classe n'étant qu'une généralisation de la notion de type, il est par
conséquent possible de définir des patrons de classe

\begin{minted}[fontsize=\footnotesize,samepage,mathescape,xrightmargin=0.5cm,xleftmargin=0.5cm]{c++}
template<typename T> class point
{
public:
  point(T abs_ = 0, T ord_ = 0) : m_x(abs_), m_y(ord_) {}
private:
  T m_x;
  T m_y;
};

int main()
{
  point<double> my_point1(5.0, 3.0);
  point<char>   my_point2('a', 'b');
}
\end{minted}
\end{frame}
\end{document}
