\documentclass[c]{beamer}
\usepackage{org-preamble}
\usepackage[cpp_teaching]{slide-style}
\usepackage{minted}
\newcommand{\inline}[1]{\mintinline[breaklines]{c++}{#1}}
\usepackage{xcolor}
\usepackage{newunicodechar}

\title{Bibliothèque standard : tableaux et quelques autres classes}
\subtitle{Partie 1}

\begin{document}

\maketitle

%--------------------------------------------------------------------

\begin{frame}[fragile]{Bibliothèque standard}

\begin{itemize}
\item \href{https://en.cppreference.com/w/cpp/container}{Conteneurs} : \inline{std::vector}, \inline{std::list}, \inline{std::array}, \inline{std::map} (=\texttt{dict} de python), \inline{std::stack}, \inline{std::queue}, itérateurs...
\item \href{https://en.cppreference.com/w/cpp/string}{Chaines de caractères} : \inline{std::string}, UTF-8, locales, \inline{std::sstream}, \inline{std::format}, expressions régulières...
\item \href{https://en.cppreference.com/w/cpp/algorithm}{Algorithmes} : \inline{std::transform()}, \inline{std::find()}, \inline{std::count()}, \inline{std::replace()}, \inline{std::reverse()}, \inline{std::sort()}, min, max, intersections, unions... et la toute nouvelle \href{https://en.cppreference.com/w/cpp/ranges}{\texttt{<ranges>}}
\item Mesure du temps et des dates : \href{https://en.cppreference.com/w/cpp/chrono}{\texttt{<chrono>}}
\item Génération pseudo-aléatoire (loi uniforme, normale, poissonienne...) : \href{https://en.cppreference.com/w/cpp/numeric/random}{\texttt{<random>}}
\item \href{https://en.cppreference.com/w/cpp/thread}{Parallélisme} : threads, verrous...
\item \href{https://en.cppreference.com/w/cpp/memory}{Pointeurs automatiques} : \inline{std::shared_ptr}, \inline{std::unique_ptr}
\item Opérations vectorisées ("à la numpy") : \href{https://en.cppreference.com/w/cpp/numeric/valarray}{\texttt{<valarray>}}
\item \href{https://en.cppreference.com/w/cpp/numeric/special_functions}{Fonctions mathématiques spéciales} (fonctions de Bessel...)
\item \href{https://en.cppreference.com/w/cpp/utility#General-purpose_utilities}{Objets utiles} : \inline{std::tuple}, \inline{std::optional}, \href{https://en.cppreference.com/w/cpp/numeric/complex}{\inline{std::complex}}...
\end{itemize}

\end{frame}

%---------------------------------------------------------------------

\begin{frame}[fragile]{Le conteneur \texttt{vector}}
 \begin{itemize}
\item La classe \texttt{vector} est proche par son utilisation du tableau défini en C

\item Avantages majeurs de la classe \texttt{vector} :

\begin{enumerate}
\item possibilité de stocker n'importe que type (uniforme) de données

\item capacité à réallouer automatiquement l'espace mémoire nécessaire lors de l'ajout d'éléments ou lors de redimensionnement $\rightarrow$ \emph{taille dynamique}

\item désallocation de mémoire automatique

\item propose un \emph{itérateur} : manipulable par les algorithmes de la bibliothèque standard (tri, somme...) et parcours avec les \emph{range-based loop}
\end{enumerate}
\end{itemize}
\end{frame}

\begin{frame}[fragile]{La classe \texttt{vector}}

\begin{minted}[fontsize=\footnotesize,samepage,mathescape,xrightmargin=0.5cm,xleftmargin=0.5cm]{c++}
#include <vector>
#include <iostream>

int main()
{
  // Déclaration d'un tableau d'entiers vide
  std::vector<int> tableau;

  // Ajout de trois éléments (et allocation mémoire si nécessaire)
  tableau.push_back(4);
  tableau.push_back(2);
  tableau.push_back(5);

  // La méthode size donne le nombre actuel d'éléments
  for (size_t i = 0; i < tableau.size(); ++i)
    std::cout << i << ": " << tableau[i] << std::endl;

  // La mémoire est automatiquement libérée ici
}
\end{minted}
\end{frame}

\begin{frame}[fragile]{La classe \texttt{vector}}

\begin{minted}[fontsize=\footnotesize,samepage,mathescape,xrightmargin=0.5cm,xleftmargin=0.5cm]{c++}
// Création d'un tableau d'entiers contenant 70,70,70,70,70
std::vector<int> tableau2(5, 70);

// Réassignation des valeurs de ce tableau
tableau2[0] = 5;
tableau2[1] = 3;
tableau2[2] = 7;
tableau2[3] = 4;
tableau2[4] = 8;

// Insertion après la position 2
tableau2.insert(tableau2.begin()+2, 42);

// Récupération du dernier élément du tableau (8)
cout << tableau2.back() << endl;

// Création par initializer list
std::vector<int> tableau3 = {4, 8, 1, -4, 5};

\end{minted}

\end{frame}

%-------------------------------------------------------------------------

\begin{frame}[fragile]{Range-based loops}

Tout conteneur présentant des \emph{itérateurs} peuvent être parcourus avec une boucle similaire au \texttt{for x in ...} de Python :

\begin{minted}[fontsize=\footnotesize,samepage,mathescape,xrightmargin=0.5cm,xleftmargin=0.5cm]{c++}
for (int x : tableau2) {
  cout << x << endl;
}
\end{minted}

Si les éléments doivent être modifiés, il faut les prendre par référence :

\begin{minted}[fontsize=\footnotesize,samepage,mathescape,xrightmargin=0.5cm,xleftmargin=0.5cm]{c++}
for (int& x : tableau2) {
  x += 2;
}
\end{minted}

\end{frame}

%-------------------------------------------------------------------------

\begin{frame}[fragile]{[Bonus] Quelques algorithmes}

Les \href{https://en.cppreference.com/w/cpp/algorithm}{algorithmes de la bibliothèque standard} sont dans le header \texttt{<algorithm>}.

\vspace{1em}
Recherche :

\begin{minted}[fontsize=\footnotesize,samepage,mathescape,xrightmargin=0.5cm,xleftmargin=0.5cm]{c++}
std::vector::iterator it = std::find(tableau3.begin(), tableau3.end(), 4);

if (it != tableau3.end())
  cout << "4 est en position" << (it - tableau3.begin()) << endl;
\end{minted}

Ici, \texttt{it} est un \emph{itérateur}. Se comportent comme un pointeur. \inline{vector::begin()} pointe le premier élément, \inline{vector::end()} pointe après le dernier élément. Peuvent être avancés (\inline{it++}, \inline{it+5}), reculés (\inline{it--}), soustraits...

\vspace{1em}
Tri ascendant :

\begin{minted}[fontsize=\footnotesize,samepage,mathescape,xrightmargin=0.5cm,xleftmargin=0.5cm]{c++}
std::sort(tableau3.begin(), tableau3.end());
// tableau3 est maintenant : {-4, 1, 4, 5, 8}
\end{minted}

\end{frame}


%-------------------------------------------------------------------------

% \begin{frame}[fragile,label={sec:orgheadline8}]{La classe \texttt{vector}}
%  \begin{minted}[fontsize=\footnotesize,samepage,mathescape,xrightmargin=0.5cm,xleftmargin=0.5cm]{c++}
% // Création d'un vecteur de particule
% std::vector<particule> my_particles;

% // Création d'un ensemble de particule
% for (size_t i = 0; i < 10; ++i) {
%   particule my_particle(0.511*i, -1.6e-19*i);
%   my_particles.push_back(my_particle);

%   // Affiche la dernière particule
%   my_particles.back().affiche();
% }
% \end{minted}
% \end{frame}

%-------------------------------------------------------------------------

\begin{frame}[fragile]{Tableau vs vector : remplissage à partir d'un fichier}

Exemple de remplissage d'un tableau de points à partir d'un fichier de points (fichier deux colonnes \texttt{x} et \texttt{y})
  
\begin{minted}[fontsize=\footnotesize,samepage,mathescape,xrightmargin=0.5cm,xleftmargin=0.5cm]{c++}
  struct point {
    double x, y;
  };
\end{minted}

\end{frame}

%------------------------------------------------------------------------------------------
\begin{frame}[fragile]{Tableau vs vector : remplissage à partir d'un fichier}

  \begin{columns}[t]
    \hspace{-0.8cm} 
    \begin{column}{0.6\textwidth}
      \begin{minted}[fontsize=\footnotesize,samepage,mathescape,xrightmargin=0.1cm,xleftmargin=0.1cm]{c++}
      ifstream data("donnees.txt");
      double x, y;
      unsigned int i = 0, nl = 0;
      while (data >> x >> y) i++;
      data.close();
      nl = i;

      point * tabpoint = new point[nl];
      i = 0;
      data.open("donnees.txt");
      while (data >> x >> y) {
        tabpoint[i] = {x, y};
        i++;
      }

      for (i = 0; i < nl; i++)
        cout << tabpoint[i].x << "," << tabpoint[i].y << endl;

      delete[] tabpoint;
      \end{minted}
    \end{column}
    \pause
    \hspace{-0.6cm} 
    \begin{column}{0.6\textwidth}
      \begin{minted}[fontsize=\footnotesize,samepage,mathescape,xrightmargin=0.1cm,xleftmargin=0.1cm]{c++}
      ifstream data("donnees.txt");
      double x, y;
      vector<point> tabpoint;
      while (data >> x >> y) {
        tabpoint.push_back({x, y});
      }

      for (point p : tabpoint)
        cout << tabpoint[i].x << ","
             << tabpoint[i].y << endl;
      \end{minted}
    \end{column}
  \end{columns}
\end{frame}

%-----------------------------------------------------------------------------

\begin{frame}[fragile]{Chaînes de caractères}
\begin{itemize}
\item En C, le texte est représenté par des chaines de caractères \inline{char*} terminées par un zéro. Comme les tableaux C, pénible à manipuler.

\item Dans le même esprit que \inline{std::vector} : la classe \href{https://en.cppreference.com/w/cpp/string}{\inline{std::string}}

\item Avantages majeurs :

\begin{enumerate}
\item capacité à réallouer automatiquement l'espace mémoire nécessaire lors de l'ajout de texte ou lors de redimensionnement $\rightarrow$ \emph{taille dynamique}

\item désallocation de mémoire automatique

\item concaténations de chaînes, insertion, recherche, sous-chaîne...
\end{enumerate}
\end{itemize}
\end{frame}

%------------------------------

\begin{frame}[fragile]{Chaînes de caractères : exemple}

\begin{minted}[fontsize=\footnotesize,samepage,mathescape,xrightmargin=0.5cm,xleftmargin=0.5cm]{c++}

// Initialisation à partir d'un char*
std::string s1 = "C'est";

// Ajout à la chaine
s1 += " bien ";

std::string s2 = "partique ?";
// Remplacement en position 1
s2.replace(1, 2, "ra");
// Accès à un caractère par indice
s2[s2.size()-1] = '!';

// Concaténation et assignation à une nouvelle std::string
auto s = s1 + s2;

std::cout << s << std::endl;
// Affiche : C'est bien pratique !

\end{minted}

\end{frame}

%------------------------------

\begin{frame}[fragile]{Chaînes de caractères : autre exemple}

\begin{minted}[fontsize=\footnotesize,samepage,mathescape,xrightmargin=0.5cm,xleftmargin=0.5cm]{c++}
using namespace std;

// Initialisation d'une chaîne vide
string s;

// Lecture depuis le terminal
cin >> s;

// Recherche du premier caractère '@'
string::size_type pos = s.find('@');
if (pos != string::npos) {
  cout << "Le caractère @ a été trouvé en position " << pos << endl;

  // Sous-chaîne : affiche tout ce qui est avant le @
  cout << s.substr(0, pos) << endl;
}

\end{minted}

\end{frame}

%------------------------------

\begin{frame}[fragile]{Conversion chaînes $\rightarrow$ nombres}

\begin{itemize}

\item \inline{std::istringstream} : fonctionne comme \inline{std::cin} ou \inline{std::ifstream}

\item \texttt{stoi}, \texttt{stol}, \texttt{stoul}, \texttt{stof}, \texttt{stod}, \texttt{stold}...

\end{itemize}

\end{frame}

%------------------------------

\begin{frame}[fragile]{[Bonus] Formattage des chaînes de caractères}

\large{\textbf{1. Avec \texttt{std::ostringstream}}}\\

Avantage : s'utilise exactement comme \texttt{std::cout}. N'affiche rien en sortie, mais stocke le texte dans un buffer. Le formattage s'effectue avec des \href{https://en.cppreference.com/w/cpp/io/manip}{\textit{manipulateurs}}.

\begin{minted}[fontsize=\footnotesize,samepage,mathescape,xrightmargin=0.5cm,xleftmargin=0.5cm]{c++}

#include <sstream>
#include <iomanip>

float pi = 3.1415;
std::ostringstream s;  // déclaration du buffer-flux
s << std::fixed << std::setprecision(2);
s << "pi = " << pi;
std::string texte = s.str();  // récupération de la chaine de caractère

\end{minted}

$\rightarrow$ Chaine de caractère \texttt{texte} contenant "\texttt{pi = 3.14}".

\end{frame}

\begin{frame}[fragile]{[Bonus] Formattage des chaînes de caractères}

\large{\textbf{2. Avec \href{https://en.cppreference.com/w/cpp/string/basic_string/to_string}{\texttt{std::to\_string}}}}\\
  
Convertit un nombre en chaine de caractère. Exemple :

\begin{minted}[fontsize=\footnotesize,samepage,mathescape,xrightmargin=0.5cm,xleftmargin=0.5cm]{c++}

#include <string>
using namespace std::string_literals;

int x = 4;
std::string texte = "chose"s + std::to_string(x) + "truc";

\end{minted}

$\rightarrow$ Chaine de caractère \texttt{texte} contenant "\texttt{chose4truc}".\\
  
\small{Peu flexible pour les flottants : notation décimale seulement et précision fixée.\\Note : l'opérateur \texttt{+} n'est défini que sur \inline{std::string}, pas sur \inline{const char*}, d'où la nécessité d'écrire \inline{std::string("chose")+...} ou le littéral \inline{"chose"s} associé et pas \inline{"chose"+...}}\\
\end{frame}

\begin{frame}[fragile]{[Bonus] Formattage des chaînes de caractères}

\large{\textbf{3. Avec \href{https://en.cppreference.com/w/cpp/utility/format}{\texttt{std::format}}}}\\
  
Ce qui est le plus familier et pratique. Fonctionne exactement comme en Python !

\begin{minted}[fontsize=\footnotesize,samepage,mathescape,xrightmargin=0.5cm,xleftmargin=0.5cm]{c++}

#include <format>

double t = 1234.5;
double x = 0.776, y = 0.921;
std::string texte =
  std::format("Position at t={:.1e} : (x={:.2f},y={:.2f})", t, x, y);

\end{minted}

$\rightarrow$ "\texttt{Position at t=1.2e+3 : (x=0.78,y=0.92)}".\\

{\small\textcolor{gray}{\textit{Attention, la version de la bibliothèque standard sur les machines du magistère n'est pas assez récente pour comporter \texttt{std::format}...} Vous pouvez installer son ancêtre, la \href{https://fmt.dev/latest/index.html}{libfmt}}.}

\end{frame}

%------------------------------

\end{document}
