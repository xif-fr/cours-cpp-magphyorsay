\documentclass[c]{beamer}
\usepackage{org-preamble}
\usepackage[cpp_teaching]{slide-style}
\usepackage{minted}
\usetheme{default}
\newcommand{\inline}[1]{\mintinline[breaklines]{c++}{#1}}


\title{Copie d'objets et ``const''}

\begin{document}

\maketitle
\setbeamercovered{invisible}

\begin{frame}[fragile]{Rappel : passage par valeur, par référence}
 \begin{columns}
\begin{column}{0.5\columnwidth}
\begin{cbox}[][lwuc][\footnotesize](Passage par valeur)
\begin{minted}[fontsize=\footnotesize,samepage,mathescape,xrightmargin=0.5cm,xleftmargin=0.5cm]{c++}
void f(int i) {
  i = 10;
  cout << "f : " << i << endl;
}

void main() {
  int e = 0;
  f(e);
  cout << "main : " << e << endl;
}
\end{minted}

\onslide<2>\tt
\begin{cbox}
f : 10\\
main : 0
\end{cbox}

\onslide<all>
\end{cbox}
\end{column}
\begin{column}{0.5\columnwidth}
\begin{cbox}[][lwuc][\footnotesize](Passage par référence)
\begin{minted}[fontsize=\footnotesize,samepage,mathescape,xrightmargin=0.5cm,xleftmargin=0.5cm]{c++}
void g(int& i) {
  i = 10;
  cout << "g : " << i << endl;
}

void main() {
  int e = 0;
  g(e);
  cout << "main : " << e << endl;
}
\end{minted}

\onslide<2>\tt
\begin{cbox}
g : 10\\
main : 10
\end{cbox}
\end{cbox}
\end{column}
\end{columns}
\end{frame}

%-----------------------------------------

\begin{frame}[fragile]{Passage d'objets par valeur, par référence}
 \begin{columns}
\begin{column}{0.5\columnwidth}
\begin{cbox}[][lwuc][\footnotesize](Passage par valeur)
\begin{minted}[fontsize=\footnotesize,samepage,mathescape,xrightmargin=0.5cm,xleftmargin=0.5cm]{c++}
void f(Complexe z) {
  z.set_real_imag(2, 3);
  cout << "f : ";
  z.affiche();
}

void main() {
  Complexe p(1, 0.5);
  f(p);
  cout << "main : ";
  p.affiche();
}
\end{minted}

\onslide<2>\tt
\begin{cbox}
f : (2 + 3i)\\
main : (1 + 0.5i)
\end{cbox}

\onslide<all>
\end{cbox}
\end{column}
\begin{column}{0.5\columnwidth}
\begin{cbox}[][lwuc][\footnotesize](Passage par référence)
\begin{minted}[fontsize=\footnotesize,samepage,mathescape,xrightmargin=0.5cm,xleftmargin=0.5cm]{c++}
void g(Complexe& z) {
  z.set_real_imag(2, 3);
  cout << "g : ";
  z.affiche();
}

void main() {
  Complexe p(1, 0.5);
  g(p);
  cout << "main : ";
  p.affiche();
}
\end{minted}

\onslide<2>\tt
\begin{cbox}
g : (2 + 3i)\\
main : (2 + 3i)
\end{cbox}
\end{cbox}
\end{column}
\end{columns}
\end{frame}

%-------------------------------------

\begin{frame}[fragile]{Passage d'objets par valeur, par référence constante}
 \onslide<all>
\begin{columns}
\begin{column}{0.5\columnwidth}
\begin{cbox}[][lwuc][\footnotesize](Passage par valeur)
\begin{minted}[fontsize=\footnotesize,samepage,mathescape,xrightmargin=0.5cm,xleftmargin=0.5cm]{c++}
void f(Complexe z) {
  cout << "Le complexe vaut ";
  z.affiche();
}

void main() {
  Complexe p(1, 0.5);
  f(p);
}
\end{minted}

\onslide<2>
\begin{cbox}[][][\centering]
L'objet est copié (constructeur par copie appelé)
\end{cbox}

\onslide<all>
\end{cbox}
\end{column}
\begin{column}{0.5\columnwidth}
\begin{cbox}[][lwuc][\footnotesize](Passage par référence constante)
\begin{minted}[fontsize=\footnotesize,samepage,mathescape,xrightmargin=0.5cm,xleftmargin=0.5cm]{c++}
void g(const Complexe& z) {
  cout << "Le complexe vaut ";
  z.affiche();
}

void main() {
  Complexe p(1, 0.5);
  g(p);
}
\end{minted}

\onslide<2>
\begin{cbox}[][][\centering]
Une copie inutile est évitée
\end{cbox}
\vspace{0.45em}
\end{cbox}
\end{column}
\end{columns}

\vspace{1em}
\pause
$\rightarrow$ Utile pour les objets volumineux (tableaux, listes...). Parfois indispensable (objet non copiable).

\end{frame}

%---------------------------------------------

\begin{frame}[fragile]{Objets constants}
Le mot-clé \inline{const} interdit la modification de l'objet auquel il s'applique.
\vspace{1em}

\begin{minted}[fontsize=\footnotesize,samepage,mathescape,xrightmargin=0.5cm,xleftmargin=0.5cm]{c++}
const int i = 42;
i = 3;  // erreur de compilation
\end{minted}

\vspace{1em}
\pause
\begin{minted}[fontsize=\footnotesize,samepage,mathescape,xrightmargin=0.5cm,xleftmargin=0.5cm]{c++}
void f(const Complexe& z)
{
  z.affiche();  // possible
  z.set_real_imag(1, 2);  // impossible
}
\end{minted}

\pause

Le compilateur doit "savoir" quelles méthodes modifient l'objet.
\end{frame}

%---------------------------------------------

\begin{frame}[fragile]{Méthodes constantes}

\def\theFancyVerbLine{%
  \color{white}\sffamily\tiny\arabic{FancyVerbLine}%
        {\tikz[remember picture,overlay]\node(minted-\arabic{FancyVerbLine}){};}%
}
\tikzset{codeblock/.style={color=#1!50,rounded corners=0.5ex, opacity=0.2, fill}}

Le prototype d'une méthode doit se terminer par \inline{const} si celle-ci ne modifie pas l'objet :
\vspace{1em}

\begin{minted}[linenos,firstnumber=1,fontsize=\footnotesize,samepage,mathescape,xrightmargin=0.5cm,xleftmargin=0.5cm]{c++}
class Complexe
{
public:
  ...
  void affiche() const;
  void norme() const;
  void set_real_imag(double real, double imag);
  ...
};
\end{minted}
\begin{tikzpicture}[remember picture,overlay]
  \draw[codeblock=blue]
  ([yshift=-0.5ex,xshift=14.75ex]minted-6) rectangle
  ([yshift=+1.25ex,xshift=19.15ex]minted-6);
  \draw[codeblock=blue]
  ([yshift=-0.5ex,xshift=16.5ex]minted-5) rectangle
  ([yshift=+1.25ex,xshift=21ex]minted-5);
\end{tikzpicture}

\pause
\begin{cbox}[6][lbtuc][\centering\small][9][5]
\ding{42} Ces méthodes ne peuvent pas modifier les membres de la classe
\end{cbox}

\begin{itemize}
\item Par défaut les méthodes peuvent modifier l'objet.
\item En cas d'oubli de \inline{const}, la méthode \emph{n'est pas utilisable sur des objets constants} !
\end{itemize}
\end{frame}

%---------------------------------------------

\begin{frame}[fragile]{Exemple de copie évitable}
Mais on peut souvent éviter des copies inutiles :

\begin{columns}
\begin{column}{0.5\columnwidth}
\begin{minted}[fontsize=\footnotesize,samepage,mathescape,xrightmargin=0.5cm,xleftmargin=0.5cm]{c++}
Complexe f()
{
  Complexe z(1, 2);
  return z;
}
\end{minted}

\begin{cbox}[][lwuc][\centering\footnotesize]
La variable locale \texttt{z} est copiée dans un objet anonyme retourné par la
fonction.
\end{cbox}

\pause
\end{column}
\begin{column}{0.5\columnwidth}
\begin{minted}[fontsize=\footnotesize,samepage,mathescape,xrightmargin=0.5cm,xleftmargin=0.5cm]{c++}
Complexe f()
{
  return Complexe(1, 2);
}
\end{minted}

\vspace{0.7em}

\begin{cbox}[][lwuc][\centering\footnotesize]
Il n'y a plus de variable locale. L'objet anonyme est construit directement.
\end{cbox}
\end{column}
\end{columns}

\vspace{1.5em}
\pause
{\small Le compilateur est souvent capable d'effectuer ces optimisations automatiquement, mais mieux vaut prendre des bonnes habitudes.}

\end{frame}



\end{document}
